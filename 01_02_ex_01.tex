\documentclass{ximera}

\usepackage{todonotes}

\newcommand{\RR}{\mathbb R}
\renewcommand{\d}{\,d}
\newcommand{\dd}[2][]{\frac{d #1}{d #2}}
\renewcommand{\l}{\ell}
\newcommand{\ddx}{\frac{d}{dx}}
\newcommand{\dfn}{\textbf}
\newcommand{\eval}[1]{\bigg[ #1 \bigg]}
\renewcommand{\epsilon}{\varepsilon}
\newcommand{\p}[1]{\left(#1\right)}
\newcommand{\br}[1]{\left[#1\right]}
\newcommand{\set}[1]{\left\{#1\right\}}


\let\prelim\lim
\renewcommand{\lim}{\displaystyle\prelim}

\colorlet{textColor}{black} 
\colorlet{background}{white}
\colorlet{penColor}{blue!50!black} % Color of a curve in a plot
\colorlet{penColor2}{red!50!black}% Color of a curve in a plot
\colorlet{penColor3}{red!50!blue} % Color of a curve in a plot
\colorlet{penColor4}{green!50!black} % Color of a curve in a plot
\colorlet{penColor5}{orange!80!black} % Color of a curve in a plot
\colorlet{fill1}{blue!50!black!20} % Color of fill in a plot
\colorlet{fill2}{blue!10} % Color of fill in a plot
\colorlet{fillp}{fill1} % Color of positive area
\colorlet{filln}{red!50!black!20} % Color of negative area
\colorlet{gridColor}{gray!50} % Color of grid in a plot


\newcommand{\fullwidth}{}
\newcommand{\normalwidth}{}



%% makes a snazzy t-chart for evaluating functions
\newenvironment{tchart}{\rowcolors{2}{}{background!90!textColor}\array}{\endarray}


\author{Gregory Hartman \and Matthew Carr}
\license{Creative Commons 3.0 By-NC}
\acknowledgement{https://github.com/APEXCalculus}

\begin{document}
\begin{exercise}

\outcome{State the precise definition of a limit.}
\outcome{Understand the concept of a limit.}

\tag{limit}
\tag{formal definition of the limit}

% I'm not sure how we give the formal definition of the limit, if we do give it.
% I've edited the source to conform to my preferred convention as I find
% statements using `whenever' to be confusing.


What is wrong with the following ``definition'' of a limit?
	\begin{quote}
``The limit of $f\p{x}$, as $x$ approaches $a$, is $L$'' means that given any $\delta>0$ there exists $\epsilon>0$ such that if $\abs{f\p{x}-L}< \epsilon$, then we have $\abs{x-a}<\delta$.
	\end{quote}
	
\begin{multipleChoice}
 \choice{Nothing, this definition is correct.}
 \choice{It should be ``given any $\epsilon>0$," not $\delta>0$.}
 \choice{It should be that if $\abs{x-a}<\delta$, then $\abs{f\p{x}-L}<\epsilon$, not the other way around.}
 \choice[correct]{It should be ``given any $\epsilon>0$," not $\delta>0$, and it be that if $\abs{x-a}<\delta$, then we have $\abs{f\p{x}-L}< \epsilon$, not the other way around.}
\end{multipleChoice}

\end{exercise}
\end{document}