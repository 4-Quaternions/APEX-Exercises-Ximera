\documentclass{ximera}

\input{preamble.tex}

\author{Gregory Hartman \and Matthew Carr}
\license{Creative Commons 3.0 By-NC}
\acknowledgement{https://github.com/APEXCalculus}

\begin{document}
\begin{exercise}

\outcome{Calculate limits using the limit laws.}
\outcome{Calculate limits using the Squeeze Theorem.}

\tag{limit} 
\tag{continuous}

% BADBAD
% I believe the author meant \lim_{x\to3^{-}} (i.e., 3^-) instead of 3^+. 
% As discussed (e.g., that we don't expect them to know that \sqrt{x} is continuous at x=0)
% I have extended the interval from [0,3] to [0,5] and let the limit be x\to3 

Find 
\[
\lim_{x\to3}f\p{x}
\begin{prompt}
= \answer{9}.
\end{prompt}
\]
given on the interval $\left[0,5\right]$ that
\[
6x-9\le f\p{x}\le x^2.
\]
\begin{hint}
Use the Squeeze Theorem on the two bounds for $f\p{x}$. Observe that $\lim_{x\to3}\p{6x-9}=6\cdot\lim_{x\to3}\p{x}-\lim_{x\to3}\p{9}=9$ and $\lim_{x\to3}\p{x^2}=\p{\lim_{x\to3}\p{x}}^2=9$.
\end{hint}
\begin{hint}
Since we are given that $6x-9\le f\p{x}\le x^2$ on $\left[0,5\right]$, and we have seen that $\lim_{x\to3}\p{6x-3}=\lim_{x\to3}\p{x^2}=9$, it follows by the Squeeze Theorem that $\lim_{x\to3}f\p{x}=9$.
\end{hint}
\end{exercise}
\end{document}