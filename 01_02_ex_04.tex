\documentclass{ximera}

\usepackage{todonotes}

\newcommand{\RR}{\mathbb R}
\renewcommand{\d}{\,d}
\newcommand{\dd}[2][]{\frac{d #1}{d #2}}
\renewcommand{\l}{\ell}
\newcommand{\ddx}{\frac{d}{dx}}
\newcommand{\dfn}{\textbf}
\newcommand{\eval}[1]{\bigg[ #1 \bigg]}
\renewcommand{\epsilon}{\varepsilon}
\newcommand{\p}[1]{\left(#1\right)}
\newcommand{\br}[1]{\left[#1\right]}
\newcommand{\set}[1]{\left\{#1\right\}}


\let\prelim\lim
\renewcommand{\lim}{\displaystyle\prelim}

\colorlet{textColor}{black} 
\colorlet{background}{white}
\colorlet{penColor}{blue!50!black} % Color of a curve in a plot
\colorlet{penColor2}{red!50!black}% Color of a curve in a plot
\colorlet{penColor3}{red!50!blue} % Color of a curve in a plot
\colorlet{penColor4}{green!50!black} % Color of a curve in a plot
\colorlet{penColor5}{orange!80!black} % Color of a curve in a plot
\colorlet{fill1}{blue!50!black!20} % Color of fill in a plot
\colorlet{fill2}{blue!10} % Color of fill in a plot
\colorlet{fillp}{fill1} % Color of positive area
\colorlet{filln}{red!50!black!20} % Color of negative area
\colorlet{gridColor}{gray!50} % Color of grid in a plot


\newcommand{\fullwidth}{}
\newcommand{\normalwidth}{}



%% makes a snazzy t-chart for evaluating functions
\newenvironment{tchart}{\rowcolors{2}{}{background!90!textColor}\array}{\endarray}


\author{Gregory Hartman \and Matthew Carr}
\license{Creative Commons 3.0 By-NC}
\acknowledgement{https://github.com/APEXCalculus}

\begin{document}
\begin{exercise}

\outcome{State the precise definition of a limit.}
\outcome{Understand the concept of a limit.}

\tag{limit}
\tag{formal definition of the limit}

Let $f(x)=3-x$. Let $\epsilon>0$ be given. Verify that
\[
\lim_{x\to5}f(x)=-2
\]
by finding a suitable choice for $\delta>0$ such that if $\left|{x-5}\right|<\delta$, then $\left|{f(x)-({-2})}\right|<\epsilon$. Choose $\delta\begin{prompt} = \answer{\epsilon}.\end{prompt}$

 \begin{hint}
 Observe that $\left|{f(x)-({-2})}\right|=\left|{3-x-({-2})}\right|$ and $\left|{3-x-({-2})}\right|=\left|{5-x}\right|$. How should we choose $\delta>0$ such that if $0<\left|{x-5}\right|<\delta$, then $\left|{5-x}\right|<\epsilon$?
 \end{hint}
 \begin{hint}
 Saying that $\left|{5-x}\right|<\epsilon$ is equivalent to saying that the quantity $\left({5-x}\right)$ satisfies $-\epsilon<\left({5-x}\right)<\epsilon$. Multiplying by a negative number reverses the direction of the inequalities (if you don't remember this, convince yourself of it).
 
 Therefore, upon multiplication of both sides of $-\epsilon<\left({5-x}\right)$ by $-1$, we have $\left({-1\cdot\left({-\epsilon}\right)}\right)>\left({-1\cdot\left({5-x}\right)}\right)$, which is simply $\epsilon>\left({x-5}\right)$. Similarly, the statement $\left({5-x}\right)<\epsilon$, upon multiplication of both sides by $-1$ becomes $\left({-1\cdot\left({5-x}\right)}\right)>\left({-1\cdot\left({\epsilon}\right)}\right)$ which is simply $\left({x-5}\right)>-\epsilon$.
 \end{hint}
 \begin{hint}
 Combining the statements that $\epsilon>\left({x-5}\right)$ and $\left({x-5}\right)>-\epsilon$, we see they say that $-\epsilon<\left({x-5}\right)<\epsilon$. Thus, if $-\epsilon<\left({5-x}\right)<\epsilon$, then $-\epsilon<\left({x-5}\right)<\epsilon$, and if $-\epsilon<\left({x-5}\right)<\epsilon$, then $-\epsilon<\left({5-x}\right)<\epsilon$. This makes sense, because, in fact, $\left|{x-5}\right|=\left|{5-x}\right|$ for any real number $x$. 
 
 Thus, we can recast the question as follows. How should we choose $\delta>0$ such that if $0<\left|{x-5}\right|<\delta$, then $\left|{x-5}\right|<\epsilon$?
 \end{hint}
 \begin{hint}
 Choose $\delta=\epsilon$. We see that this choice gives us the desired result. Namely, if $0<\left|{x-5}\right|<\epsilon$, then $\left|{5-x}\right|<\epsilon$. This follows because, from what we have shown, if $\left|{x-5}\right|<\epsilon$, then $\left|{5-x}\right|<\epsilon$.
 \end{hint}
\end{exercise}
\end{document}