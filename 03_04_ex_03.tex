\documentclass{ximera}

\usepackage{todonotes}

\newcommand{\RR}{\mathbb R}
\renewcommand{\d}{\,d}
\newcommand{\dd}[2][]{\frac{d #1}{d #2}}
\renewcommand{\l}{\ell}
\newcommand{\ddx}{\frac{d}{dx}}
\newcommand{\dfn}{\textbf}
\newcommand{\eval}[1]{\bigg[ #1 \bigg]}
\renewcommand{\epsilon}{\varepsilon}
\newcommand{\p}[1]{\left(#1\right)}
\newcommand{\br}[1]{\left[#1\right]}
\newcommand{\set}[1]{\left\{#1\right\}}


\let\prelim\lim
\renewcommand{\lim}{\displaystyle\prelim}

\colorlet{textColor}{black} 
\colorlet{background}{white}
\colorlet{penColor}{blue!50!black} % Color of a curve in a plot
\colorlet{penColor2}{red!50!black}% Color of a curve in a plot
\colorlet{penColor3}{red!50!blue} % Color of a curve in a plot
\colorlet{penColor4}{green!50!black} % Color of a curve in a plot
\colorlet{penColor5}{orange!80!black} % Color of a curve in a plot
\colorlet{fill1}{blue!50!black!20} % Color of fill in a plot
\colorlet{fill2}{blue!10} % Color of fill in a plot
\colorlet{fillp}{fill1} % Color of positive area
\colorlet{filln}{red!50!black!20} % Color of negative area
\colorlet{gridColor}{gray!50} % Color of grid in a plot


\newcommand{\fullwidth}{}
\newcommand{\normalwidth}{}



%% makes a snazzy t-chart for evaluating functions
\newenvironment{tchart}{\rowcolors{2}{}{background!90!textColor}\array}{\endarray}


\author{Gregory Hartman \and Matthew Carr}
\license{Creative Commons 3.0 By-NC}
\acknowledgement{https://github.com/APEXCalculus}

\begin{document}
\begin{exercise}

\outcome{Identify the relationships between the function and its first and second derivatives.}

%% BADBAD
%% Original question did not include 'twice differentiable'
%% I included this so to make the relation between the function
%% And it's first and second derivative clear
%% Also, I'm not aware if they know the definition
%% for concave and convex function in general

\tag{concavity}
\tag{derivative}

True or False?
\begin{quote}
It is possible for a twice differentiable function to be \textbf{increasing} and \textbf{concave down} on $(0,\infty)$ with a horizontal asymptote of $y=1$.
\end{quote}
\begin{prompt}
\begin{multipleChoice}
\choice[correct]{True}
\choice{False}
\end{multipleChoice}
\end{prompt}

\begin{hint}
Consider the graph of $f(x)=\frac{2}{\pi}\arctan(x)$. This should either provide a confirmation of your suspicion or a counter example. Notice that $\left|f(x)\right|\le1$, so $f(x)$ has a horizontal asymptote at $y=1$.
\end{hint}
\begin{hint}
$f(x)$ is increasing. Recalling the derivative of the inverse tangent function, $f'(x)=\frac{1}{1+x^2}$, so $f'(x)>0$ for all $x$, in particular, for all $x$ in the interval $(0,\infty)$, so $f(x)$ is increasing on $(0,\infty)$.
\end{hint}

\begin{hint}
$f(x)$ is concave down. By the chain rule and the quotient rule, $f''(x)=-\frac{2x}{(1+x^2)^2}$ which is less than $0$ \emph{precisely} on $(0,\infty)$. So $f(x)$ is concave down.

Thus, $f$ is an example which demonstrates that it is possible for a function to be increasing (i.e., the first derivative is positive) and 
\end{hint}

\end{exercise}
\end{document}