\documentclass{ximera}

\usepackage{todonotes}

\newcommand{\RR}{\mathbb R}
\renewcommand{\d}{\,d}
\newcommand{\dd}[2][]{\frac{d #1}{d #2}}
\renewcommand{\l}{\ell}
\newcommand{\ddx}{\frac{d}{dx}}
\newcommand{\dfn}{\textbf}
\newcommand{\eval}[1]{\bigg[ #1 \bigg]}
\renewcommand{\epsilon}{\varepsilon}
\newcommand{\p}[1]{\left(#1\right)}
\newcommand{\br}[1]{\left[#1\right]}
\newcommand{\set}[1]{\left\{#1\right\}}


\let\prelim\lim
\renewcommand{\lim}{\displaystyle\prelim}

\colorlet{textColor}{black} 
\colorlet{background}{white}
\colorlet{penColor}{blue!50!black} % Color of a curve in a plot
\colorlet{penColor2}{red!50!black}% Color of a curve in a plot
\colorlet{penColor3}{red!50!blue} % Color of a curve in a plot
\colorlet{penColor4}{green!50!black} % Color of a curve in a plot
\colorlet{penColor5}{orange!80!black} % Color of a curve in a plot
\colorlet{fill1}{blue!50!black!20} % Color of fill in a plot
\colorlet{fill2}{blue!10} % Color of fill in a plot
\colorlet{fillp}{fill1} % Color of positive area
\colorlet{filln}{red!50!black!20} % Color of negative area
\colorlet{gridColor}{gray!50} % Color of grid in a plot


\newcommand{\fullwidth}{}
\newcommand{\normalwidth}{}



%% makes a snazzy t-chart for evaluating functions
\newenvironment{tchart}{\rowcolors{2}{}{background!90!textColor}\array}{\endarray}


\author{Gregory Hartman \and Matthew Carr}
\license{Creative Commons 3.0 By-NC}
\acknowledgement{https://github.com/APEXCalculus}

\begin{document}
\begin{exercise}

\outcome{Estimate limits using nearby values.}
\outcome{Estimating limits numerically and possible errors to this method.}

\tag{limit}
\tag{derivative}

% The r@{.}l aligns everything at a common decimal point by constructing two columns and 
% aligning them together at that decimal point to create the effect that only a single column is
% there. It's a quick fix to a very minor blemish in the source.

% We need the multicolumn{2}{c} to align the 'h' over the center of the two columns we have     
% made, since we cannot just change the number of columns midway through our construction. 
% By indicating that there are 2 columns we are merging, TeX understands our construction.

% I have to  put extra columns into this one, else the RHS doesn't line up quite right. Better way to 
% do this?



Let $f\p{x} =\ln\p{x}$ and $a=5$. Observe the table of values for $\frac{f\p{a+h}-f\p{a}}{h}$:
\begin{center}
 \begin{tabular}{r@{.}lc@{\hspace{30pt}}r@{.}l}
  \multicolumn{2}{c}{$h$} & \multicolumn{3}{c}{$\frac{f\p{a+h}-f\p{a}}{h}$}\\ \hline 
  $-0$ & $1$ & & $0$ & $202027$  \\
  $-0$ & $01$ & & $0$ & $2002$ \\
  $0$ & $01$ & & $0$ & $1998$ \\
  $0$ & $1$ & & $0$ & $198026$
 \end{tabular}
\end{center}
The limit as $h\to 0$ is approximately $\begin{prompt}\answer{0.2}\end{prompt}$.

\end{exercise}
\end{document}