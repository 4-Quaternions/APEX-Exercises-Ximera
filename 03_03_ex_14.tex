\documentclass{ximera}

\usepackage{todonotes}

\newcommand{\RR}{\mathbb R}
\renewcommand{\d}{\,d}
\newcommand{\dd}[2][]{\frac{d #1}{d #2}}
\renewcommand{\l}{\ell}
\newcommand{\ddx}{\frac{d}{dx}}
\newcommand{\dfn}{\textbf}
\newcommand{\eval}[1]{\bigg[ #1 \bigg]}
\renewcommand{\epsilon}{\varepsilon}
\newcommand{\p}[1]{\left(#1\right)}
\newcommand{\br}[1]{\left[#1\right]}
\newcommand{\set}[1]{\left\{#1\right\}}


\let\prelim\lim
\renewcommand{\lim}{\displaystyle\prelim}

\colorlet{textColor}{black} 
\colorlet{background}{white}
\colorlet{penColor}{blue!50!black} % Color of a curve in a plot
\colorlet{penColor2}{red!50!black}% Color of a curve in a plot
\colorlet{penColor3}{red!50!blue} % Color of a curve in a plot
\colorlet{penColor4}{green!50!black} % Color of a curve in a plot
\colorlet{penColor5}{orange!80!black} % Color of a curve in a plot
\colorlet{fill1}{blue!50!black!20} % Color of fill in a plot
\colorlet{fill2}{blue!10} % Color of fill in a plot
\colorlet{fillp}{fill1} % Color of positive area
\colorlet{filln}{red!50!black!20} % Color of negative area
\colorlet{gridColor}{gray!50} % Color of grid in a plot


\newcommand{\fullwidth}{}
\newcommand{\normalwidth}{}



%% makes a snazzy t-chart for evaluating functions
\newenvironment{tchart}{\rowcolors{2}{}{background!90!textColor}\array}{\endarray}


\author{Gregory Hartman \and Matthew Carr}
\license{Creative Commons 3.0 By-NC}
\acknowledgement{https://github.com/APEXCalculus}

\begin{document}
\begin{exercise}

\outcome{Use the first derivative to determine whether a function is increasing or decreasing.}
\outcome{Understand what information the derivative gives concerning when a function is increasing or decreasing.}
\outcome{Find domain and range.}
\outcome{Find critical points.}
\outcome{Find all local maximums and minimums using the 1st and 2nd derivative tests.}

\tag{increasing}
\tag{decreasing}
\tag{derivative test}

Let $f(x)=x^2+2x-3$. 
\begin{enumerate}
\item		The domain of $f$ is $\begin{prompt}\answer{(-\infty,\infty)}\end{prompt}$.
\item		$f$ has a critical point(s) at $x\begin{prompt} = \answer{-1}\end{prompt}$.
\item		$f$ is decreasing on $\begin{prompt}\answer{(-\infty,1)}\end{prompt}$.
\item		$f$ is increasing on $\begin{prompt}\answer{(-1,\infty)}\end{prompt}$.
\item		At $x=-1$, $f$ has
\begin{prompt}
\begin{multipleChoice}
\choice[correct]{A local minimum}
\choice{A local maximum}
\choice{Both a local maximum and a local minimum}
\choice{Neither a local maximum nor a local minimum}
\end{multipleChoice}
\end{prompt}
\end{enumerate}

\end{exercise}
\end{document}