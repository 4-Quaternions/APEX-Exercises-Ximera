\documentclass{ximera}

\usepackage{todonotes}

\newcommand{\RR}{\mathbb R}
\renewcommand{\d}{\,d}
\newcommand{\dd}[2][]{\frac{d #1}{d #2}}
\renewcommand{\l}{\ell}
\newcommand{\ddx}{\frac{d}{dx}}
\newcommand{\dfn}{\textbf}
\newcommand{\eval}[1]{\bigg[ #1 \bigg]}
\renewcommand{\epsilon}{\varepsilon}
\newcommand{\p}[1]{\left(#1\right)}
\newcommand{\br}[1]{\left[#1\right]}
\newcommand{\set}[1]{\left\{#1\right\}}


\let\prelim\lim
\renewcommand{\lim}{\displaystyle\prelim}

\colorlet{textColor}{black} 
\colorlet{background}{white}
\colorlet{penColor}{blue!50!black} % Color of a curve in a plot
\colorlet{penColor2}{red!50!black}% Color of a curve in a plot
\colorlet{penColor3}{red!50!blue} % Color of a curve in a plot
\colorlet{penColor4}{green!50!black} % Color of a curve in a plot
\colorlet{penColor5}{orange!80!black} % Color of a curve in a plot
\colorlet{fill1}{blue!50!black!20} % Color of fill in a plot
\colorlet{fill2}{blue!10} % Color of fill in a plot
\colorlet{fillp}{fill1} % Color of positive area
\colorlet{filln}{red!50!black!20} % Color of negative area
\colorlet{gridColor}{gray!50} % Color of grid in a plot


\newcommand{\fullwidth}{}
\newcommand{\normalwidth}{}



%% makes a snazzy t-chart for evaluating functions
\newenvironment{tchart}{\rowcolors{2}{}{background!90!textColor}\array}{\endarray}


\author{Gregory Hartman \and Matthew Carr}
\license{Creative Commons 3.0 By-NC}
\acknowledgement{https://github.com/APEXCalculus}

\begin{document}
\begin{exercise}

\outcome{Calculate limits using the limit laws.}
\outcome{Calculate limits of the form 0/0.}
\outcome{Calculate limits using the Squeeze Theorem.}

\tag{limit} 
\tag{indeterminate form}
\tag{discontinuous}

% BADBAD 

It is a well known fact that $\lim_{x\to\infty}\left({\left({1+\frac{1}{x}}\right)^{x}}\right)=\lim_{x\to0}\left({(1+x)^{1/x}}\right)=e$.
Find 
\[
\lim_{x\to0}\left({\frac{\ln(1+x)}{x}}\right)
\begin{prompt}
= \answer{1}.
\end{prompt}
\]


\begin{hint}
Recall that $\lim_{x\to\infty}\left({\left({1+\frac{1}{x}}\right)^{x}}\right)=e$ and, equivalently, $\lim_{x\to0}\left({(1+x)^{1/x}}\right)=e$. Observe that we can rewrite this in a more illuminating way as $\frac{\ln(1+x)}{x}=x^{-1}\ln(1+x)$. Apply some properties of logarithms to this.
\end{hint}
\begin{hint}
We know, from the properties of logarithms, that $b\ln(a)=\ln({a^b})$ when $a>0$. Hence, $x^{-1}\ln(1+x)=\ln({(1+x)^{1/x}})$ for $x>-1$ (and, of course, $x^{-1}=\frac{1}{x}$). Hence, $\ln({(1+x)^{1/x}})=\frac{\ln(1+x)}{x}$ when $x\ne0$. So $\lim_{x\to0}\left(\ln({(1+x)^{1/x}})\right)=\lim_{x\to0}\left(\frac{\ln(1+x)}{x}\right)$. Recall that $\ln(x)$ is continuous everywhere besides $x=0$. What can you say about $\lim_{x\to0}\ln({(1+x)^{1/x}})$?
\end{hint}
\begin{hint}
Let $y=(1+x)^{1/x}$. Then, since $\lim_{x\to0}\left({(1+x)^{1/x}}\right)=e$, and $\ln(x)$ is continuous at $e$, it follows that $\lim_{x\to0}\left({\ln({(1+x)^{1/x}}})\right)=\lim_{y\to e}\left({\ln(y)}\right)=1$ and, hence, that $\lim_{x\to0}\left({\frac{\ln(1+x)}{x}}\right)=1$.
\end{hint}
\end{exercise}
\end{document}