\documentclass{ximera}

\usepackage{todonotes}

\newcommand{\RR}{\mathbb R}
\renewcommand{\d}{\,d}
\newcommand{\dd}[2][]{\frac{d #1}{d #2}}
\renewcommand{\l}{\ell}
\newcommand{\ddx}{\frac{d}{dx}}
\newcommand{\dfn}{\textbf}
\newcommand{\eval}[1]{\bigg[ #1 \bigg]}
\renewcommand{\epsilon}{\varepsilon}
\newcommand{\p}[1]{\left(#1\right)}
\newcommand{\br}[1]{\left[#1\right]}
\newcommand{\set}[1]{\left\{#1\right\}}


\let\prelim\lim
\renewcommand{\lim}{\displaystyle\prelim}

\colorlet{textColor}{black} 
\colorlet{background}{white}
\colorlet{penColor}{blue!50!black} % Color of a curve in a plot
\colorlet{penColor2}{red!50!black}% Color of a curve in a plot
\colorlet{penColor3}{red!50!blue} % Color of a curve in a plot
\colorlet{penColor4}{green!50!black} % Color of a curve in a plot
\colorlet{penColor5}{orange!80!black} % Color of a curve in a plot
\colorlet{fill1}{blue!50!black!20} % Color of fill in a plot
\colorlet{fill2}{blue!10} % Color of fill in a plot
\colorlet{fillp}{fill1} % Color of positive area
\colorlet{filln}{red!50!black!20} % Color of negative area
\colorlet{gridColor}{gray!50} % Color of grid in a plot


\newcommand{\fullwidth}{}
\newcommand{\normalwidth}{}



%% makes a snazzy t-chart for evaluating functions
\newenvironment{tchart}{\rowcolors{2}{}{background!90!textColor}\array}{\endarray}


\author{Matthew Carr}
%% BADBAD
%% From https://math.osu.edu/sites/math.osu.edu/files/Math1151_SampleMidterm1_SP15.pdf
%% License?
%% Acknowledgement?
\license{BADBAD}
\acknowledgement{BADBAD}

\begin{document}
\begin{exercise}

\outcome{Calculate limits using the limit laws.}
\outcome{Calculate limits of the form 0/0.}

\tag{limits}

Find
\[
\lim_{x\to5}\left(\frac{\sqrt{x-4}-1}{x-5}\right)
\begin{prompt}
= \answer{6}
\end{prompt}
\]

\begin{hint}
Multiply the numerator and denominator by the conjugate of $\sqrt{x-4}-1$, namely, $\sqrt{x-4}+1$.
\end{hint}
\begin{hint}
$(\sqrt{x-4}-1)(\sqrt{x-4}+1)=x-4-1=x-5$. So $\frac{\sqrt{x-4}-1}{x-5}\cdot\frac{\sqrt{x-4}+1}{\sqrt{x-4}+1}=\frac{x-5}{(x-5)(\sqrt{x-4}+1)}=\frac{1}{\sqrt{x-4}+1}$.
\end{hint}
\begin{hint}
Then $\lim_{x\to5}\frac{1}{\sqrt{x-4}+1}=\frac{1}{2}$.
\end{hint}

\end{exercise}
\end{document}