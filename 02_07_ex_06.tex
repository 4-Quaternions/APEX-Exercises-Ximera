\documentclass{ximera}

\usepackage{todonotes}

\newcommand{\RR}{\mathbb R}
\renewcommand{\d}{\,d}
\newcommand{\dd}[2][]{\frac{d #1}{d #2}}
\renewcommand{\l}{\ell}
\newcommand{\ddx}{\frac{d}{dx}}
\newcommand{\dfn}{\textbf}
\newcommand{\eval}[1]{\bigg[ #1 \bigg]}
\renewcommand{\epsilon}{\varepsilon}
\newcommand{\p}[1]{\left(#1\right)}
\newcommand{\br}[1]{\left[#1\right]}
\newcommand{\set}[1]{\left\{#1\right\}}


\let\prelim\lim
\renewcommand{\lim}{\displaystyle\prelim}

\colorlet{textColor}{black} 
\colorlet{background}{white}
\colorlet{penColor}{blue!50!black} % Color of a curve in a plot
\colorlet{penColor2}{red!50!black}% Color of a curve in a plot
\colorlet{penColor3}{red!50!blue} % Color of a curve in a plot
\colorlet{penColor4}{green!50!black} % Color of a curve in a plot
\colorlet{penColor5}{orange!80!black} % Color of a curve in a plot
\colorlet{fill1}{blue!50!black!20} % Color of fill in a plot
\colorlet{fill2}{blue!10} % Color of fill in a plot
\colorlet{fillp}{fill1} % Color of positive area
\colorlet{filln}{red!50!black!20} % Color of negative area
\colorlet{gridColor}{gray!50} % Color of grid in a plot


\newcommand{\fullwidth}{}
\newcommand{\normalwidth}{}



%% makes a snazzy t-chart for evaluating functions
\newenvironment{tchart}{\rowcolors{2}{}{background!90!textColor}\array}{\endarray}


\author{Gregory Hartman \and Matthew Carr}
\license{Creative Commons 3.0 By-NC}
\acknowledgement{https://github.com/APEXCalculus}

\begin{document}
\begin{exercise}

\tag{inverse function}

\outcome{Define and work with inverse functions.}

Let $f(x)=x^2+6x+11$ be defined for all $x\ge3$. A calculus student proposes the function $g(x)=\sqrt{x-2}-3$, defined for all $x\ge2$, to be the inverse of $f$. Which of the following is true?

\begin{prompt}
\begin{multipleChoice}

\choice{$g(f(x))=x$ for all $x\ge3$ and $f(g(x))=x$ for all $x\ge2$. Hence, $g$ is the inverse of $f$.}

\choice[correct]{$g(f(x))=x$ for all $x\ge{3}$ but $f(g(x))=x$ only for all $x\ge{38}$. Hence, $g$ is not the inverse of $f$.}

\choice{$g(f(x))=x$ for all $x\ge2$ but $f(g(x))=x$ only for all $x\ge{38}$. Hence, $g$ is not the inverse of $f$.}

\choice{Since both $g(f(x))$ and $f(g(x))$ are not equal to $1$, $g$ is not the inverse of $f$.}

\end{multipleChoice}
\end{prompt}


\end{exercise}
\end{document}