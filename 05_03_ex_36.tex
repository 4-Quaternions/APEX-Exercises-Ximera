\documentclass{ximera}

\usepackage{todonotes}

\newcommand{\RR}{\mathbb R}
\renewcommand{\d}{\,d}
\newcommand{\dd}[2][]{\frac{d #1}{d #2}}
\renewcommand{\l}{\ell}
\newcommand{\ddx}{\frac{d}{dx}}
\newcommand{\dfn}{\textbf}
\newcommand{\eval}[1]{\bigg[ #1 \bigg]}
\renewcommand{\epsilon}{\varepsilon}
\newcommand{\p}[1]{\left(#1\right)}
\newcommand{\br}[1]{\left[#1\right]}
\newcommand{\set}[1]{\left\{#1\right\}}


\let\prelim\lim
\renewcommand{\lim}{\displaystyle\prelim}

\colorlet{textColor}{black} 
\colorlet{background}{white}
\colorlet{penColor}{blue!50!black} % Color of a curve in a plot
\colorlet{penColor2}{red!50!black}% Color of a curve in a plot
\colorlet{penColor3}{red!50!blue} % Color of a curve in a plot
\colorlet{penColor4}{green!50!black} % Color of a curve in a plot
\colorlet{penColor5}{orange!80!black} % Color of a curve in a plot
\colorlet{fill1}{blue!50!black!20} % Color of fill in a plot
\colorlet{fill2}{blue!10} % Color of fill in a plot
\colorlet{fillp}{fill1} % Color of positive area
\colorlet{filln}{red!50!black!20} % Color of negative area
\colorlet{gridColor}{gray!50} % Color of grid in a plot


\newcommand{\fullwidth}{}
\newcommand{\normalwidth}{}



%% makes a snazzy t-chart for evaluating functions
\newenvironment{tchart}{\rowcolors{2}{}{background!90!textColor}\array}{\endarray}


\author{Gregory Hartman \and Matthew Carr}
\license{Creative Commons 3.0 By-NC}
\acknowledgement{https://github.com/APEXCalculus}

\begin{document}
\begin{exercise}

\outcome{Add up a large number of terms quickly using sigma notation.}
\outcome{Compute left, right, and midpoint Riemann Sums with many rectangles.}
\outcome{Understand the relationship between area under a curve and sums of rectangles.}
\outcome{Approximate area under a curve.}
\outcome{Understand how the area under a curve is related to the antiderivative.}

%% BADBAD
%% Original Author has answers as 20/3 but I can't get that
%% I've done it the correct (I hope) way

\tag{sum}
\tag{Riemann sum}
\tag{integral}

Answer the following, given that it is a well known fact that  $\sum_{k=1}^{m}k^2=\frac{m(m+1)(2m+1)}{6}$ and $\sum_{k=1}^{m}k=\frac{m(m+1)}{2}$.
\begin{enumerate}
\item		Find the left Riemann Sum for $n$ equally spaced rectangles approximating $\int_{1}^{4}2x^2-3\d x$ \[\int_{1}^{4}2x^2-3\d x\approx\answer{33+\frac{9}{n^2}-\frac{45}{n}}\]
\item		The limit as $n\to\infty$ of your answer to part (a) is: \[\answer{33}\]
\item		Does the answer to part (b) agree with the value of the integral $\int_{1}^{4}2x^2-3\d x$? 
\begin{multipleChoice}
\choice[correct]{Yes}
\choice{No}
\end{multipleChoice}
\end{enumerate}
\end{exercise}
\end{document}