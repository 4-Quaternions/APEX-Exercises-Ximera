\documentclass{ximera}

\input{preamble.tex}

\author{Gregory Hartman \and Matthew Carr}
\license{Creative Commons 3.0 By-NC}
\acknowledgement{https://github.com/APEXCalculus}

\begin{document}
\begin{exercise}

\outcome{Use the first derivative to determine whether a function is increasing or decreasing.}
\outcome{Understand what information the derivative gives concerning when a function is increasing or decreasing.}
\outcome{Find domain and range.}
\outcome{Find critical points.}
\outcome{Find all local maximums and minimums using the 1st and 2nd derivative tests.}

%% BADBAD
%% f has a critical point(s) at x=DNE
%% DNE answer
%% Is that OK?


\tag{increasing}
\tag{decreasing}
\tag{derivative test}

Let $f(x)=\frac{x}{x^2-2x-8}$. 
\begin{enumerate}
\item		The domain of $f$ is $\begin{prompt}\answer{(-\infty,-2)\cup(-2,4)\cup(4,\infty)}\end{prompt}$.
\item		$f$ has a critical point(s) at $x\begin{prompt} = \answer{DNE}\end{prompt}$.
\item		$f$ is decreasing on $\begin{prompt}\answer{(-\infty,-2)\cup(-2,4)\cup(4,\infty)}\end{prompt}$.
\item		$f$ is increasing on $\begin{prompt}\answer{DNE}\end{prompt}$.
\item		At $x=0$, $f$ has 
\begin{prompt}
\begin{multipleChoice}
\choice{A local minimum}
\choice{A local maximum}
\choice{Both a local maximum and a local minimum}
\choice[correct]{Neither a local maximum nor a local minimum}
\end{multipleChoice}
\end{prompt}
\end{enumerate}

\end{exercise}
\end{document}