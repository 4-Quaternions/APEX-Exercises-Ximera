\documentclass{ximera}

\usepackage{todonotes}

\newcommand{\RR}{\mathbb R}
\renewcommand{\d}{\,d}
\newcommand{\dd}[2][]{\frac{d #1}{d #2}}
\renewcommand{\l}{\ell}
\newcommand{\ddx}{\frac{d}{dx}}
\newcommand{\dfn}{\textbf}
\newcommand{\eval}[1]{\bigg[ #1 \bigg]}
\renewcommand{\epsilon}{\varepsilon}
\newcommand{\p}[1]{\left(#1\right)}
\newcommand{\br}[1]{\left[#1\right]}
\newcommand{\set}[1]{\left\{#1\right\}}


\let\prelim\lim
\renewcommand{\lim}{\displaystyle\prelim}

\colorlet{textColor}{black} 
\colorlet{background}{white}
\colorlet{penColor}{blue!50!black} % Color of a curve in a plot
\colorlet{penColor2}{red!50!black}% Color of a curve in a plot
\colorlet{penColor3}{red!50!blue} % Color of a curve in a plot
\colorlet{penColor4}{green!50!black} % Color of a curve in a plot
\colorlet{penColor5}{orange!80!black} % Color of a curve in a plot
\colorlet{fill1}{blue!50!black!20} % Color of fill in a plot
\colorlet{fill2}{blue!10} % Color of fill in a plot
\colorlet{fillp}{fill1} % Color of positive area
\colorlet{filln}{red!50!black!20} % Color of negative area
\colorlet{gridColor}{gray!50} % Color of grid in a plot


\newcommand{\fullwidth}{}
\newcommand{\normalwidth}{}



%% makes a snazzy t-chart for evaluating functions
\newenvironment{tchart}{\rowcolors{2}{}{background!90!textColor}\array}{\endarray}


\author{Gregory Hartman \and Matthew Carr}
\license{Creative Commons 3.0 By-NC}
\acknowledgement{https://github.com/APEXCalculus}

\begin{document}
\begin{exercise}

\outcome{Solve basic related rates word problems.}
\outcome{Identify word problems as related rates problems.}
\outcome{Understand the process of solving related rates problems.}

\tag{inflection points}
\tag{critical points}
\tag{derivative test}

%% BADBAD
%% Format OK?

A circular balloon is inflated with air flowing at a rate of $10$cm$^3$/s. How fast is the radius of the balloon increasing when the radius is: \begin{enumerate}
\item		$1$cm? \[\answer{\frac{5}{2\pi}}\,cm/s\]
\item		$10$cm? \[\answer{\frac{1}{40\pi}}\,cm/s\]
\item		$100$cm? \[\answer{\frac{1}{4000\pi}}\,cm/s\]

\end{enumerate}

\begin{hint}
The volume of the balloon as a function of time is $V(t)=\frac{4}{3}\pi r^3$ where $r$ is the radius of the balloon as a function of time. What is the derivative of $V(t)$ with respect to $t$. What about the derivative of $r^2$ with respect to $t$, given that $r$ is a function of time?
\end{hint}
\begin{hint}
The derivative as a function of time is $10=4\pi r^2\frac{dr}{dt}$ since $\frac{dV}{dt}=10$cm$^3$/s.
\end{hint}
\begin{hint}
Then $\frac{dr}{dt}=\frac{5}{2r^2\pi}$. For any value of $r$, we can find $\frac{dr}{dt}$ by plugging in that value.
\end{hint}
\end{exercise}
\end{document}